\documentclass[a4paper,12pt]{article}

% PACOTES
\usepackage[portuguese]{babel}
\usepackage[utf8]{inputenc}
\usepackage[T1]{fontenc}
\usepackage[a4paper,top=2cm,bottom=2cm,left=3cm,right=3cm]{geometry}
\usepackage{graphicx}          % para inserir imagens
\usepackage{amsmath}           % símbolos matemáticos
\usepackage[colorlinks=true,
            linkcolor=red,        % cor dos links internos (ex: índice, seções)
            urlcolor=blue,         % cor dos links externos (URLs)
            citecolor=green]{hyperref} % cor das referências (se tiveres citações) 
\usepackage{listings}          % para inserir código
\usepackage{xcolor}            % cores no código
\usepackage[numbered,framed]{matlab-prettifier} % codigo no matlab
\usepackage{indentfirst}

\lstset{
    style=Matlab-editor, % Usa o estilo base
    literate=
     {á}{{\'a}}1 {à}{{\`a}}1 {ã}{{\~a}}1 {â}{{\^a}}1 {Á}{{\'A}}1 {À}{{\`A}}1 {Ã}{{\~A}}1 {Â}{{\^A}}1
     {é}{{\'e}}1 {è}{{\`e}}1 {ê}{{\^e}}1 {É}{{\'E}}1 {È}{{\`E}}1 {Ê}{{\^E}}1
     {í}{{\'i}}1 {ì}{{\`i}}1 {î}{{\^i}}1 {Í}{{\'I}}1 {Ì}{{\`I}}1 {Î}{{\^I}}1
     {ó}{{\'o}}1 {ò}{{\`o}}1 {õ}{{\~o}}1 {ô}{{\^o}}1 {Ó}{{\'O}}1 {Ò}{{\`O}}1 {Õ}{{\~O}}1 {Ô}{{\^O}}1
     {ú}{{\'u}}1 {ù}{{\`u}}1 {û}{{\^u}}1 {Ú}{{\'U}}1 {Ù}{{\`U}}1 {Û}{{\^U}}1
     {ç}{{\c{c}}}1 {Ç}{{\c{C}}}1,
    extendedchars=true % Garante que os caracteres são processados
}

% INFORMAÇÕES DO TRABALHO
\title{\textbf{Trabalho de Introdução à Robótica}}
\author{Tomás Espincho -- 110486}
\date{1 de novembro de 2025}

\setlength{\parindent}{1.25cm}

\begin{document}

% CAPA
\begin{titlepage}
    \centering

\noindent
    \hfill
    \includegraphics[width=0.30\textwidth]{LogoUA.png}

    
    {\Huge \textbf{Trabalho de Introdução à Robótica}}\\[1cm]
    {\large Licenciatura em Automação Industrial}\\[2cm]

    \includegraphics{Captura de ecrã 2025-11-01 233349.png}
    
    \vfill

    {\Large Tomás Espincho (110486)}\\[0.5cm]

    {\large  1 de novembro de 2025}\\[2cm]

\end{titlepage}

% INÍCIO DO TRABALHO
\newpage
\tableofcontents
\newpage

\section{Introdução}

Neste projeto, o objetivo consiste em utilizar o \textit{MATLAB} para simular o robô \textbf{UR10} a realizar uma tarefa de manipulação. A simulação envolve o robô a pegar em dois blocos e a colocá-los sobre uma mesa, mantendo uma distância de 50\,cm entre eles.

O robô inicia o movimento a partir da posição vertical (\textit{home}), pega num dos blocos e posiciona-o sobre a bancada com redundância do cotovelo para cima e movimento por juntas, calculado através da cinemática inversa (\href{https://www.petercorke.com/RTB/r9/html/SerialLink.html#ikine}{\textit{SerialLink.ikine}}) e a posição da ponta através da cinemática direta {(\href{https://www.petercorke.com/RTB/r9/html/SerialLink.html#fkine}{\textit{SerialLink.fkine}})}. Em seguida, retorna à posição inicial (\textit{home}) e repete o processo com o segundo bloco, desta vez com a redundância do cotovelo para baixo. Após posicionar ambos os blocos, o robô utiliza a sua ferramenta para desenhar uma linha reta sobre a bancada, movimento por juntas, calculado através do Jacobiano (\href{https://www.petercorke.com/RTB/r9/html/SerialLink.html#jacob0}{\textit{SerialLink.jacob0}}), a ligar  dois vértices dos blocos.

\section{Explicação do Código}
O código \textbf{MATLAB} utilizado neste trabalho é apresentado e explicado nas subsecções seguintes.

\subsection{Função Principal}
A função principal é responsável por inicializar as variáveis e executar o programa.

\begin{lstlisting}[style = Matlab-editor, caption = {Função principal, e iniciação de parametros}]
function tp1_110486(offset)

%% Parametros Iniciais

if nargin < 0.01
    offset = 0; % valor por omissão
end

%% Parametros Base do Robo

% Carregar Robo

mdl_ur10

% Criar a ponta (linha de 15 cm no eixo z do último elo)

TamanhoPonta = 0.15;
ur10.tool = transl(0, 0, TamanhoPonta);

Home =[-pi/2 -pi/2 0 -pi/2 -pi/2 0]; % Posição Home
ur10.base = transl(0, 0, 0); % Situar a Base no (0,0,0)
\end{lstlisting}


\subsection{Partes da Função Principal}

\subsubsection{Movimento dos blocos por Cinemática inversa}
\begin{enumerate}
    \item \textbf{Definição das Posições:} São definidas as matrizes de transformação para os pontos: uma posição de aproximação segura acima do bloco, a posição exata de "pegar" (\textit{pick}), uma posição de segurança para mover o bloco, e a posição final de "largar" (\textit{place}). Todas as posições incluem uma rotação de 180º em torno de X para a ferramenta apontar para baixo.
    
    \item \textbf{Cálculo da Cinemática Inversa:} Para cada posição (ex: \texttt{MPickVrd}), as juntas correspondente ao calculo com a função \texttt{ur10.ikine(MPickVrd)}.
    
    \item \textbf{Geração da Trajetória:} Com as posições de juntas (ex: \texttt{Home} e \texttt{JPickVrd}), a função \texttt{jtraj(Home, JPickVrd, NSteps)} é usada para criar a trajetória.
    
    \item \textbf{Animação e Atualização do Bloco:} O robô é animado por \texttt{ur10.animate()} ou \texttt{ur10.plot()} dentro de um ciclo \texttt{for}. Durante os movimentos em que o bloco está "agarrado", a função auxiliar \texttt{AtualizaBloco} é chamada para redesenhar os vértices do bloco, fazendo-o seguir o robô.
\end{enumerate}

\subsubsection{Desenho da linha através do jacobiano}
\begin{enumerate}
    \item \textbf{Definição da Trajetória Cartesiana:} Calculo da trajetória em 3D através do \texttt{ctraj}, com o ponto inicial (\texttt{MStart}) e final (\texttt{MFinish}).

    \item \textbf{Cálculo da velocidade das juntas:} Calculo da velocidades cartesianas através da função \texttt{tr2delta}.

    \item \textbf{Jacobiano:} Calculo do jacobiano, através do \texttt{jacob0}, as velocidades lineares e angulares das juntas pelo jacobiano inverso através do \texttt{pinv}.
\end{enumerate}

É implementado pelo seguinte ciclo \texttt{for}: 
\begin{lstlisting}[style = Matlab-editor, caption = {Ciclo for para implementar o jacobiano}]
for k = 1:(NSteps - 1)
    Pos = TrajLine(k, :); % Posição atual das juntas
    J = ur10.jacob0(Pos); % Obter o Jacobiano (6x6) para a configuração atual
    Ji = pinv(J);
    dq = Ji * DeltaPath(:, k);
    TrajLine(k+1, :) = Pos + dq';
end
\end{lstlisting}

\subsection{Funções Auxiliares}
Complementam o funcionamento da função principal.

\begin{lstlisting}[style = Matlab-editor, caption={Função auxiliar usada para Criar Blocos e a "mesa"}]
function B = CriaBloco(Tipo, Pos, Cor, Dimensoes)
(...)
end

\end{lstlisting}

Função usa a função \textbf{Patch} do MATLAB para criar os blocos, através dos vértices e das faces do cubo/retângulo. 

\begin{lstlisting}[style = Matlab-editor, caption={Função auxiliar para atualizar a posição do bloco}]
function AtualizaBloco(Bloco,traj,Robo,i)
(...)
end
\end{lstlisting}

Esta função usa o i de um ciclo for, para receber a posição anterior do bloco e atualiza-o para a próxima posição usando cinemática direta.
\section{Vídeo do Youtube}

\begin{center}
\href{https://youtu.be/Y7lUG2VKP8M}{Vídeo No Youtube}
\end{center}

\section{Referências e Links Úteis}

\begin{itemize}
    \item \textbf{Documentação MATLAB:} \href{https://www.mathworks.com/help/matlab/}{\url{https://www.mathworks.com/help/matlab/}}
    \item \textbf{Documentação do SerialLink Peter Corke:} 
    
    \href{https://www.petercorke.com/RTB/r9/html/SerialLink.html}{\url{https://www.petercorke.com/RTB/r9/html/SerialLink.html}}
    
    \item \textbf{Jacobian matrix and determinant Wikipedia:} 
    
    \href{https://en.wikipedia.org/wiki/Jacobian_matrix_and_determinant}{\url{https://en.wikipedia.org/wiki/Jacobian_matrix_and_determinant}}

    \item \textbf{Resposta no Matlab Help Central :} 
    
    \href{https://www.mathworks.com/matlabcentral/answers/1578550-how-to-plot-a-3d-cube-based-if-i-have-the-coordinates-of-the-8-surrounding-nodes#answer_823944}{\url{https://www.mathworks.com/matlabcentral/answers/1578550-how-to-plot-a-3d-cube-based-if-i-have-the-coordinates-of-the-8-surrounding-nodes#answer_823944}}

    \item \textbf{Prompts usados em LLM's} (Não estão em ordem cronológica)
    \begin{itemize}
        \item \textbf{Interação 1:} Prompt do Utilizador: "Estou com dificuldades a meter acentos e tiles e ç na parte de lstlisting como resolvo"
        \item \textbf{Interação 2:} Prompt do Utilizador: "Como faço um linespace em 3d matlab ?"
        \item \textbf{Interação 3:} Prompt do Utilizador: "Quando atualizo o bloco, como faço para ele ficar na ponta dos 15 cm ?"
        \item \textbf{Interação 4:} Prompt do Utilizador: "Quando uso o patch é normal deixar as faces translucidas ? Se sim não há forma de as deixar completamente opacas ?"
        \item \textbf{Interação 5:} Prompt do Utilizador: "O plot esta muito grande como faço para dar plot inclinado mais pequeno"
        \item \textbf{Interação 6 (Revisão):} O LLM foi usado para rever a lógica do código MATLAB (incluindo a função \texttt{AtualizaBloco} e a implementação do Jacobiano) e para ajustar o código LaTeX (resolvendo erros de compilação e acentuação).
    \end{itemize}
\end{itemize}

\end{document}